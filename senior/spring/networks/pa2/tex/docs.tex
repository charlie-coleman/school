\documentclass{article}

\usepackage{amsmath}
\usepackage{amssymb}
\usepackage[margin=1in]{geometry}
\usepackage{listings}
\usepackage{xcolor}
\usepackage{varwidth}

\definecolor{deepblue}{rgb}{0,0,0.5}
\definecolor{deepred}{rgb}{0.6,0,0}
\definecolor{deepgreen}{rgb}{0,0.5,0}

\lstdefinestyle{Python}{
    language        = Python,
    basicstyle      = \ttfamily\small,
    keywordstyle    = \color{deepblue},
    keywordstyle    = [2] \color{teal}, % just to check that it works
    stringstyle     = \color{deepgreen},
    commentstyle    = \color{deepred}\ttfamily,
    frame       = single,
    numbers     = left,
    showspaces  = false,
    showstringspaces    = false,
    captionpos  = t,
    caption     = \lstname
}

\renewcommand{\lstlistingname}{Program}

\title{Documentation for CSCI 3650 Programming Assignment \#2}
\author{Charlie Coleman}

\begin{document}

	\maketitle

	\section{Documentation}

	\subsection{Configuration to Network Description}
	
	Firstly, you need to create a configuration file for your network. Some examples are given in the \texttt{./configs/} directory, but if you'd like to make your own the format is as follows.
	
	\begin{verbatim}
		nodes: Int
		topology: linear|full|star|random
		alpha: Float
		node-min: Int
		node-max: Int
		link-min: Int
		link-max: Int
	\end{verbatim}
	
	Once you have a properly formatted configuration file, you can execute the code with the following command:
	
	\begin{center}
	\verb|python3 part1.py <file in> <file out>|
	\end{center}	
	
	Only the input file name is required. If \texttt{file out} is left out, it will default to the name of the input with \texttt{.out} as the extension. This code will output a file of the format:
	
	\begin{verbatim}
	Source-Node-ID	Destination Node-ID	Link0-weight
	Source-Node-ID	Destination Node-ID	Link1-weight
	...
	Node0-weight	Node1-weight...	NodeN-weight
	\end{verbatim}

	\subsection{Network Description to Virtual Network}
	
	Once we have a network description file from \texttt{part1.py}, we can pass it to \texttt{part2.py} to get create our virtual network. To run \texttt{part2.py}, use the following command.
	
	\begin{center}
	\verb|python3 part2.py <file in>|
	\end{center}
	
	This will parse the network description file you specify and create a virtual network that matches that topology.
	
	\pagebreak
	
	\section{Code}
	
	\lstinputlisting[breaklines,style=Python]{../part1.py}
	\lstinputlisting[breaklines,style=Python]{../part2.py}
\end{document}