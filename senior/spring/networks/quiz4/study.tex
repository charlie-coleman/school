\documentclass{article}

\usepackage[margin=1in]{geometry}
\usepackage{amsmath}
\usepackage{amssymb}
\usepackage{enumitem}

\newcommand{\topic}[1]{\noindent\textbf{#1}}
\newcommand{\define}[1]{\noindent\textit{#1 |}}

\setlist{nolistsep, noitemsep}

\begin{document}

\begin{center}
	\begin{large} \textbf{Computer Networks Quiz \# 4} \end{large}
\end{center}

\hfill Charlie Coleman

\setcounter{section}{5}
\section{Link layer and LANS}

\subsection{Introduction, Services}

\topic{Terminology:}

\begin{itemize}
	\item hosts \& routers: nodes
	\item communication channels that connect adjacent nodes along communication path: links
		\begin{itemize}
			\item wired links
			\item wireless links
			\item LANs
		\end{itemize}
	\item layer-2 packet: frame, encapsulates datagram
\end{itemize}~\\

\noindent data-link layer has responsibility of transferring datagram from one node to physically adjacent node over a link.

\subsubsection*{Link Layer Services}

\topic{Framing:}

Used in almost all link-layer protocols, framing encapsulates the network-layer datagrams within a link-layer frame before transmission. Frame consists of: data field (network layer datagram is inserted here) and a number of header fields. Structure is defined by the link-layer protocol.

\topic{Link Access:}

\define{MAC} medium access control

A MAC protocol specifies the rules by which a frame is transmitted onto the link. The MAC protocol complexity is tied to how many senders/receivers exist on the line.

\topic{Reliable delivery:}

Reliable delivery means the service guarantees the transmission of the network-layer datagram without error, like TCP. Can be achieved with acks + retransmission. Reliable LL transport is typically used on links with high error rates (wireless). This prevents end-to-end retransmission. Adds potentially unnecessary overhead to low error rate conns, like fiber/ethernet.

\topic{Error detection \& correction}

Include some form of error checking bits, perform the check locally on the LL to prevent sending a bad datagram

\subsubsection*{Where is the LL implemented?}

In every host.

Typically in an 'adaptor' (network interface card) or chip (e.g. ethernet card, 802.11 card, etc.)

\subsection{Error detection, Correction}

Most basic form: Redundancy

\subsubsection*{Error Detection and Correction (EDC)}

\define{EDC} Error Detection and Correction bits, a.k.a. the redundancy

\define{D} Data protected by error checking, may include the header fields

Error Detection is not 100\% reliable!
\begin{itemize}[noitemsep,nolistsep]
	\item protocols can miss errors (rarely though)
	\item larger EDC fields == better error detection/correction (but more overhead)
\end{itemize}

\subsubsection*{How do we detect errors?}

\begin{enumerate}[noitemsep, nolistsep]
	\item Parity checking
	\begin{itemize}[noitemsep, nolistsep]
		\item Vertical parity checking
		\item Horizontal parity checking
		\item Both
	\end{itemize}
	\item Checksum (on IPv6 headers \& UDP/TCP data)
	\item CRCs
\end{enumerate}

\subsection{Multiple Access Protocols}
	
\subsubsection*{Multiple Access Links}

\topic{Two Types of "Links"}
\begin{itemize}
	\item point-to-point
	\begin{itemize}
		\item PPP for dial-up access
		\item point-to-point link between ethernet switch, host
	\end{itemize}
	\item broadcast (shared wire or medium)
	\begin{itemize}
		\item old-fashioned ethernet
		\item upstream HFC
		\item 802.11 wireless LAN
	\end{itemize}
\end{itemize}

\subsubsection*{Access Control Point Goals}

\begin{enumerate}
	\item At any time, only one needs to access the channel
	\item Even if more access the channel, there should be minimal or no collision!
\end{enumerate}

\subsubsection*{No out-of-band channel are allowed for coordination}

A "collision" occurs if a node receives two or more signals at the same time

\define{Multiple Access protocol} distributed algo that determines how nodes share a channel, i.e. determine when nodes can transmit

\define{No-out-of-bound channel constraint} communication about channel sharing must use the channel itself

\subsubsection*{An Ideal Multi-access protocol}

Given: A broadcast channel of rate R bps

\noindent Want:
\begin{enumerate}
	\item when one node wants to transmit, it can send at rate R.
	\item when M nodes want to transmit, each can send at average rate R/M
	\item fully decentralized (or distributed):
		\begin{itemize}
			\item no special node coordinate transmissions
			\item no synch of clocks, slots
		\end{itemize}
	\item As simple as possible to implement (always true)
\end{enumerate}

\subsubsection*{3 Broad Classes of MAC protocols}

\begin{itemize}
	\item Channel Partitioning
	\begin{itemize}
		\item divide channel into smaller pieces (time slots, frequency, code)
		\item allocate piece to node for exclusive use
	\end{itemize}
	\item Random Access
	\begin{itemize}
		\item channel not divided, allow collisions
		\item "recover" from collisions 
	\end{itemize}		
	\item "Taking Turns"
	\begin{itemize}
		\item nodes take turns, but nodes with more to send can take longer
	\end{itemize}
\end{itemize}

\subsubsection*{Channel Partitioning MAC Protocols}

\begin{itemize}
	\item TDMA: Time Division Multiple Access
	\begin{itemize}
		\item Access to channel in "rounds"
		\item Each situation gets fixed length slot (length = packet transmission time) in each round
		\item unused slots go idle (ew)
	\end{itemize}
	\begin{tabular}{l|l}
	Pros & Cons \\ \hline
	No Collision! & channel is underutilized when few sources have packets \\
	Fair allocation & nodes must wait their turn, even if it is free
	\end{tabular}
	
	\item FDMA: Frequency Division Multiple Access
	\begin{itemize}
		\item channel spectrum divided into frequency bands
		\item each station assigned fixed frequency band
		\item unused transmission time in frequency bands goes idle
		\item Guard bands are introduced to avoid interference
	\end{itemize}
	
	\item CDMA: Code Division Multiple Access
	\begin{itemize}
		\item Assign a different code to each channel
		\item Channels then use a different code to reconstruct bits
		\item \textbf{Multiple nodes can send at the same time!}
	\end{itemize}
	
	\item WDMA: Wavelength Division ... (used in fiber)
	\begin{itemize}
		\item Same principle as FDMA
	\end{itemize}
	
	\item SDMA: Space Division ...
	\begin{itemize}
		\item Reuse the same frequency but at different space (like radio stations)
	\end{itemize}
	
	\item OFDMA: Orthogonal Frequency Division ...
	\begin{itemize}
		\item Frequency slots are orthogonal so no guard bands
	\end{itemize}
\end{itemize}

\subsubsection*{Random Access Protocols}

When a node has a packet to send
\begin{itemize}
	\item transmit at full channel data rate R
	\item no \textit{a priori} coordination among nodes
\end{itemize}

\noindent Collision? Wait a bit (random amount) and retry!

\begin{enumerate}
	\item Slotted ALOHA (retransmit with prob p)
	
	Assumptions:
	\begin{enumerate}
		\item All frames same size
		\item Time divided into equal size slots (time to transmit 1 frame)
		\item Nodes start to transmit only slot beginning
		\item Nodes are sync'ed
		\item if $\geq$2 nodes transmit in slot, all nodes detect collision
		\item if collision, probability p you transmit in each frame until success
	\end{enumerate}
	
	\begin{itemize}
		\item Pros:
		\begin{itemize}
			\item single active node can continuously transmit at full channel rate
			\item highly decentralized: only slots in nodes need to be in sync
			\item simple
		\end{itemize}
		\item Cons:
		\begin{itemize}
			\item collisions, wasting time slots
			\item idle slots
			\item nodes may be able to detect collision in less than time to transmit packet
			\item clock sync required
		\end{itemize}
	\end{itemize}
	
	\item ALOHA
	\begin{itemize}
		\item No sync, no time slots, just wait a random amount of time
		\item Less efficient than slotted!
	\end{itemize}
	
	\item CSMA: Carrier Sense Multiple Access
	\begin{itemize}
		\item Listen before you transmit!
		\item If idle, transmit everything
		\item If busy, wait
		\item If collision, sucks for you. just do it again?
		\item propagation delay $\rightarrow$ collisions
	\end{itemize}
	
	\item CSMA/CD: CSMA + collision detection
	\begin{itemize}
		\item Cancel transmissions when collision is detected
		\item easy in wired transmission, difficult wirelessly
	\end{itemize}
\end{enumerate}

\subsubsection*{Taking Turns: Literally Jesus?}

Channel Partitioning Protocols:
\begin{itemize}
	\item share channel efficiently \& fairly at high loads
	\item super shit at low load
\end{itemize}

\noindent Random Access Protocols:
\begin{itemize}
	\item efficient at low loads
	\item 8 car pile up at high loads
\end{itemize}

\noindent "Taking Turns" Protocols:

Everything you've ever wanted and more

\noindent Examples:
\begin{itemize}
	\item Polling:
	\begin{itemize}
		\item master node invites slave nodes to transmit in turn
		\item typically used with "dumb" slave devices
		\item Concerns:
		\begin{itemize}
			\item polling overhead
			\item latency
			\item single point of failure (master)
		\end{itemize}
	\end{itemize}
	\item Token Passing
	\begin{itemize}
		\item Control token passed from one node to next sequentially
		\item token message
		\item concerns:
		\begin{itemize}
			\item token overhead
			\item latency
			\item token is a single point of failure
		\end{itemize}
	\end{itemize}
\end{itemize}

\subsection{LANs}

\subsubsection*{Addressing}

Every adapter on LAN has it's own unique LAN address

\noindent MAC addresses are allocated by IEEE, bought buy manufacturer to ensure uniqueness

\noindent MAC is portable, as LAN cards can move between LANs

\noindent IP hierarchical address is not portable, depends on IP subnet

\subsubsection*{ARP}

\define{ARP} Address Resolution Protocol, each IP node on LAN has table
\begin{itemize}
	\item IP/MAC address mappings for some LAN nodes \verb|< IP address; MAC address; TTL >|
	\item TTL (Time to Live): time after which address mappings will be forgotten (typically 20 min)
\end{itemize}

\noindent The Protocol
\begin{itemize}
	\item A wants to send datagram to B
	
	but B's MAC address not in A's ARP table
	
	\item A broadcasts ARP query packet, containing B's IP address
	\begin{itemize}
		\item destination MAC address = FF-FF-FF-FF-FF-FF
		\item all nodes on LAN receive ARP query
	\end{itemize}
	\item B receives ARP packet, replies to A with its (B's) MAC address, send to A's MAC
	\item A caches (saves) IP-to-MAC address pair in its ARP table until information becomes old (times out) (this is a soft state, information goes away)
	
	\item ARP is "plug-and-play", nodes create their own ARP table
\end{itemize}

\subsubsection*{Ethernet}

\begin{itemize}
	\item Dominant wired LAN tech
	\item Single chip, multiple speeds
	\item first widely used LAN tech
	\item simple, cheap
	\item kept up with speed race: 10Mbps - 10Gbps
	\item Connectionless - no handshaking
	\item Unreliable - no acks/nacks
	\item Uses CSMA/CD with binary back-off
\end{itemize}

\noindent Topologies
\begin{itemize}
	\item Bus - popular through mid-90s
	\begin{itemize}
		\item all nodes in same collision domain (can collide with each other)
	\end{itemize}
	\item Star - popular today
	\begin{itemize}
		\item active switch in the center of the network
		\item each spoke runs a separate ethernet protocol (no collision)
	\end{itemize}
\end{itemize}

\noindent Ethernet Frame
\begin{itemize}
	\item sending adaptar encapsulates IP datagram in Ethernet Frame
\end{itemize}
\begin{enumerate}
	\item Preamble - 7 bytes of 10101010 followed by 10101011, syncs the clock \& rate
	\item Addresses - 6 byte source \& dest MAC addresses, if it's for the adapter it passes to the network layer protocol
	\item type - indicates higher layer protocol (mostly IP, but others possible)
	\item CRC - error detection, frame dropped if error detected
\end{enumerate}

\subsubsection*{Switches}

\begin{itemize}
	\item Link-Layer device - active role
	\begin{itemize}
		\item store \& forward Ethernet frames
		\item Examine MAC address, selectively forward, use CSMA/CD
	\end{itemize}
	\item Transparent, hosts are unaware they exist
	\item Plug-and-play, self-learning | No config required!
	\item Buffer packets
	\item Separate instance of protocol on each link, no collisions
\end{itemize}

\subsubsection*{VLANs}

\define{VLAN} Virtual LAN, simplify LAN topology using software 'switches'
	
\subsection{Link Virtualization}

\subsubsection*{MPLS}

\define{MPLS} MultiProtocol Layer Switching

Operates between OSI Layer 2 and Layer 3

\begin{itemize}
	\item Initial Goal | high-speed IP forwarding using fixed length label (instead of IP address)
	\begin{itemize}
		\item fast lookup using fixed length identifier (rather than shortest prefix matching)
		\item borrowing ideas from Virtual Circuit approach
		\item but IP datagram still keeps IP address!
	\end{itemize}
	\item MPLS forwarding can differ from IP
	\begin{itemize}
		\item use destination and source addresses to route flows to same destination differently (traffic engineering)
		\item re-route flows quickly if link fails: pre-computed backup paths (useful for VoIP)
	\end{itemize}
\end{itemize}

\setcounter{section}{7}
\section{Network Security}

\subsection{What is Networking Security?}

Types of attacks
\begin{itemize}
	\item Lying
	\begin{itemize}
		\item DNS poisoning
		\item BGP false route advertising
		\item False IP or MAC address
	\end{itemize}
	\item Talking too much
	\begin{itemize}
		\item Congest someone's server
		\item Congest someone's network
	\end{itemize}
	\item Listening too much
	\begin{itemize}
		\item side channel attacks in VM
		\item Man in the middle attack
	\end{itemize}
\end{itemize}

\noindent Properties of a Secure Communication
\begin{enumerate}
	\item Confidentiality | only sender \& intended receiver should "understand" message contents
	\begin{itemize}
		\item sender encrypts message
		\item receiver decrypts message
	\end{itemize}
	
	\item End Point Authentication
	\begin{itemize}
		\item sender, receiver want to confirm identity of each other
	\end{itemize}
	
	\item Message Integrity | sender, receiver want to ensure message not altered without detection
	
	\item Access \& Availability | services must be accessible and available to users
\end{enumerate}

\subsection{Principles of Cryptography}

\begin{itemize}
	\item Symmetric Key Crypto | sender, receiver share a secret key
	\begin{itemize}
		\item Stream ciphers | have a keystream, plaintext digits are encrypted one at a time to create a stream of encrypted data
		\item Block ciphers | encrypt 1 block at a time, send it all
		\item Cipher block chaining | prevents memorizing encrypted data and comparing to future, add some randomness to the cipher and send the randomness in plaintext
		
		to save bandwidth, send once then use a generator
		
		\item \textbf{DES: Data Encryption Standard} | popular symmetric method
		\begin{itemize}
			\item US encryption standard
			\item 56 bit encryption key, 64 bit block size
			\item block cipher + CBC
			\item Not that secure, so just do it 3 times with 3 keys, duh
		\end{itemize}
		
		\item \textbf{AES: Advanced Encryption Standard} | DES? Never heard of him.
		\begin{itemize}
			\item still symmetric, replaced DES
			\item 128bit block, 128/192/256 bit keys
			\item if DES can be cracked in 1 sec, AES will take 149 trillion years
		\end{itemize}
	\end{itemize}
	
	\item Public Key Crypto | sender, receiver have a public + private key
	\begin{itemize}
		\item No symmetric key
		\item Need a function that returns the same value when executed on either side
		\item \textbf{RSA} | Rivest, Shamir, Adelson algo
		\begin{enumerate}
			\item Choose 2 large prime numbers p, q (1024+ bits ea)
			\item n = pq, z = (p-1)(q-1)
			\item e < n \&\& e is relatively prime with z
			\item choose d where e*d mod z = 1
			\item Public Key: (n, e), Private Key: (n, d)
		\end{enumerate}
		Encryption/Decryption
		\begin{enumerate}
			\item To encrypt message m, $c = m^e \text{ mod } n$
			\item To decrypt encrypted message c, $m = c^d \text{ mod } n$
		\end{enumerate}
		\begin{itemize}
			\item DES is 100x faster
			\item To improve speeds, establish a connection using RSA and generate a symmetric key, then use AES/DES/another symm key crypto
		\end{itemize}
	\end{itemize}
\end{itemize}

\subsection{Message Integrity}

\subsubsection*{Goals}

\begin{itemize}
	\item message is indeed send by intended sender
	\item message is not tampered with on its way to the receiver
\end{itemize}

\subsubsection*{Digital Signatures}

\begin{itemize}
	\item Cryptographic Hash
	\begin{itemize}
		\item Hash function (H(m)) takes an input \& computes a fixed-size string (checksum/CRC)
		\item Hashes have the property that H(x) == H(y) is very unlikely (unfeasible)
		\item Simple digital signature - encrypt with your private key!
		\item Hash functions:
		\begin{itemize}
			\item MD5 | 128 bit hash
			\item SHA-1 | 160 bit message digest
		\end{itemize}
		\item Use a secret shared key to prevent tampering with the message
	\end{itemize}
	
	\item Certification Authorities
	\begin{itemize}
		\item binds public key to a particular entity
		\item entity registers its public key with CA
		\begin{itemize}
			\item entity provides "proof of identity" to CA
			\item CA creates certificate binding entity to its public key
			\item certificate contains entity's public key digitally signed by the CA
		\end{itemize}
	\end{itemize}
\end{itemize}

\subsection{End Point Authentication}

Authenticating a conversation at the time it is occurring

\begin{itemize}
	\item needs to run before communication can commence
	\item basically use public key encryption + CA + a nonce (one time use random number to prevent replay attacks)
\end{itemize}

\subsection{Securing e-mail}

\subsubsection*{Confidentiality}

\begin{enumerate}
	\item Generate random symmetric private key
	\item encrypt message with symm priv key
	\item encrypts symm priv key with receiver's public key
	\item send both to receiver
	\item receiver decodes symm priv key using their private key
	\item receiver decodes message with symm priv key
\end{enumerate}

\subsubsection*{Message Integrity}

\begin{enumerate}
	\item Sign message with hash, m' = H(m + s)
	\item Sign with priv key
	\item Send message and encrypted hash
	\item receiver can decode the hash, hash the received message and compare
\end{enumerate}

\subsubsection*{Both?}

Chain message integrity into confidentiality

\subsection{Securing TCP Connections: SSL}

\begin{itemize}
	\item supported by almost all browsers \& web servers
	\item Base for https
	\item billions sent over SSL every year
	\item TLS - variation
	\item Provides:
	\begin{itemize}
		\item confidentiality
		\item integrity
		\item authentication
	\end{itemize}
	\item PGP (the email stuff) would not work for byte streams or interactive data
\end{itemize}

\subsubsection*{4 Components}

\begin{enumerate}
	\item Handshake | use certificates \& private keys to authenticate each other and exchange shared secret
	\item Key derivation | use shared secret to derive a set of keys
	\item Data transfer | break up data to be transferred into a series of records
	\item Connection closure | special messages to securely close connection
\end{enumerate}

\subsubsection*{Splitting into Records}

\begin{itemize}
	\item If we used streams, no message integrity until completed transfer
	\item With records, we are susceptible to replay attacks
	\item to prevent that, we use a sequence number in our hash
	\item attacker could still replay a whole connection
	\item to prevent, we use a nonce for each connection lifetime
	\item Attacker could close the connection before one side knows
	\item to prevent, add a type to the hash
\end{itemize}

\subsection{Network Layer Security: IPsec}

\subsubsection*{What is network layer confidentiality?}

\begin{itemize}
	\item Sender encrypts all network layer payload, IP payload would be hidden
	\item Payload hidden from 3rd party sniffing
	\item Said to provide "blanket coverage"
\end{itemize}

\noindent\textbf{IPsec provides confidentiality, auth, integrity, and replay-attack prevention}

\subsubsection*{Transport v Tunnel}

\begin{itemize}
	\item Transport mode:
	\begin{itemize}
		\item provides a secure connection between two endpoints as it encapsulates IP's payload (only encrypts payload)
	\end{itemize}
	\item Tunneling mode:
	\begin{itemize}
		\item encapsulates entire IP packet to provide a virtual "secure hop" between two gateways
	\end{itemize}
\end{itemize}

\subsubsection*{Security Associations \& their Databases}

\define{Security Association} simplex, info is housed at the receiver. Includes:
\begin{itemize}
	\item 32 bit SA id | Security Parameter Index (SPI)
	\item SA origin | IP addr
	\item SA dest | IP addr
	\item Encryption type
	\item Encryption key
	\item Type of Integrity check
	\item auth key
\end{itemize}

\define{SAD} Security Ass. Database, holds all that shit I just listed, key is the SPI

\subsection{Securing Wireless LANs}

\subsection{Operational Security: Firewall \& IDS}

\end{document}