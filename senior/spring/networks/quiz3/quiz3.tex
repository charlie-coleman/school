\documentclass{article}

\usepackage{amssymb}
\usepackage{amsmath}
\usepackage[margin=1in]{geometry}
\usepackage{enumitem}

\begin{document}
	\section{Network Layer}
	
	\subsection{Data Plane}
	
	\begin{itemize}[noitemsep]
		\item local, per-router function
		\item determines how datagram arriving on router input port is forwarded to router output port
		\item forwarding function
	\end{itemize}
	
	\subsection{Control Plane}
	
	\begin{itemize}[noitemsep]
		\item network-wide logic
		\item determines how datagram is routed among routers along end-end path from source host to destination host
	\end{itemize}
	
	\section{Inside a Router}
	
	\subsection{Routing Processor}
	
	performs control and management plane functions
	
	\begin{itemize}[noitemsep]
		\item maintains routing table
		\item attaches link state information to routing table
		\item computes forwarding table
		\item in SDN, it is also responsible for the communication and event handling with the 'controller'
		\item performs network management functions (e.g. packet counting stats, etc.)
	\end{itemize}
	
	\subsection{Input port functions}
	
	Line termination - physical layer functions = bit-level reception
	Link Layer Protocol - Data link layer functions = ethernet interoperation
	Lookup, forwarding, queueing - decentralized switching = lookup functions
	
	\subsection{How much to buffer?}
	
	$B = RTT * C$
	
	RTT - round trip time (e.g. 250 ms)
	
	C - Link Capacity (e.g. 10 Gbps)
	
	New eq: $B = \cfrac{RTT * C}{\sqrt{N}}$
	
	N - \# of flows
	
	\section{Internet Protocol}
	
	\subsection{Why does IP have a checksum?}
	
	\begin{enumerate}[noitemsep]
		\item IP does not have to run over TCP/UDP
		\item TCP/UDP checksum entire package, IP only header
		\item Note that checksum has to be recomputed at every router since the TTL changes
	\end{enumerate}
	
	\subsection{IPv4 Addressing}
	
	IP Address: 32-bit identifier for host, router interface
	
	\noindent Interface: connection between host/router and physical link
	\begin{itemize}[noitemsep, nolistsep]
		\item router's typically have multiple interfaces
		\item host typically has one or two interfaces (wired Ethernet, wireless 802.11)
	\end{itemize}
	
	\noindent CIDR - Classless InterDomain Routing
	\begin{itemize}[noitemsep]
		\item subnet portion of address of arbitrary length
		\item address format: a.b.c.d/x where x is the \# of bits in the subnet portion of the address
	\end{itemize}
	
	\section{Generalized Forward and SDN}
\end{document}