\documentclass[12pt]{article}

\usepackage{amsmath, amssymb}
\usepackage[margin=1in]{geometry}
\usepackage{listings}
\usepackage{enumitem}

\begin{document}
	\begin{center}
		\begin{large} Mobile Robotics Exam \# 1 \end{large}
	\end{center}
	
	\hfill Charlie Coleman
	
	\begin{enumerate}
		\item ~
		
		\begin{tabular}{|c|ccc|}
			\hline
			& SW1 & SW2 & Castor \\ \hline
			$\alpha$ & 30 & 150 & 270 \\
			$\beta$ & 0 & 0 & 0 \\
			$\gamma$ & 0 & 0 & N/A \\
			d & N/A & N/A & 1.5 \\
			l & 3.464 & 3.464 & 2 \\
			r & 2 & 2 & 1 \\ \hline
		\end{tabular}
		
		\[
			\begin{bmatrix}
				J_1 (\beta_s) \\
				C_1 (\beta_s)
			\end{bmatrix} R(\theta) \dot{\varepsilon_I} = \begin{bmatrix}
				J_2 \phi \\
				0
			\end{bmatrix}
		\]
		\[
			\begin{bmatrix}
				\begin{bmatrix}
					\sin(\alpha + \beta_c(t)) & -\cos(\alpha + \beta_c(t)) & -l_c\cos(\alpha + \beta_c(t)) \\
					\sin(30) & -\cos(30) & -l_{S1}\cos(30) \\
					\sin(150) & -\cos(150) & -l_{S2}\cos(150)
				\end{bmatrix}\\
				\begin{bmatrix}
					\cos(\alpha + \beta_c(t)) & \sin(\alpha + \beta_c(t)) & l_c \sin(\alpha + \beta_c(t)) \\
					\cos(30) & \sin(30) & l_{S1}\sin(0) \\
					\cos(150) & \sin(150) & l_{S2}\sin(0) \\
				\end{bmatrix}
			\end{bmatrix} R(\theta) \dot{\varepsilon}_I = \begin{bmatrix}
				\begin{bmatrix}
					r_c \dot{\phi_c} \\
					r_{S1} \dot{\phi}_{S1} \\
					r_{S2} \dot{\phi}_{S2}
				\end{bmatrix} \\
				\begin{bmatrix}
					d\dot{\beta_c} \\
					r_{sw} \dot{\phi}_{sw1} \\
					r_{sw} \dot{\phi}_{sw2}
				\end{bmatrix}
			\end{bmatrix}
		\]
	
	We can remove rows 1, 5, \& 6 as they do not imply any constraints on this system.
	
	\[
		\begin{bmatrix}
			\sin(30) & -\cos(30) & -l_{S1}\cos(30) \\
			\sin(150) & -\cos(150) & -l_{S2}\cos(150) \\
			\cos(\alpha + \beta_c(t)) & \sin(\alpha + \beta_c(t)) & l_c \sin(\alpha + \beta_c(t))
		\end{bmatrix} R(\theta) \dot{\varepsilon}_I = \begin{bmatrix}
			r_{S1} \dot{\phi}_{S1} \\
			r_{S2} \dot{\phi}_{S2} \\
			d\dot{\beta}_c \\
		\end{bmatrix}
	\]
	\[
		\begin{bmatrix}
			0.5 & -0.866 & -3.464 \\
			0.5 & 0.866 & -3.464 \\
			\cos(\alpha + \beta_c(t)) & \sin(\alpha + \beta_c(t)) & 2\sin(\alpha + \beta_c(t))
		\end{bmatrix} R(\theta) \dot{\varepsilon}_I = \begin{bmatrix}
			2 \dot{\phi}_{S1} \\
			2 \dot{\phi}_{S2} \\
			1.5 \dot{\beta}_c \\
		\end{bmatrix}
	\]
	
	\item $
		\dot{\varepsilon}_I = R(\theta)^{-1} \begin{bmatrix}
			0.5 & -0.866 & -3 \\
			0.5 & 0.866 & 3 \\
			\cos(270) & \sin(270) & 2\sin(270)
		\end{bmatrix}^{-1} \begin{bmatrix}
			2*1.25 \\
			2*1.2 \\
			0
		\end{bmatrix} = \begin{bmatrix}
			4.9 \text{ m/s} \\
			-0.0789 \text{ m/s} \\
			0.0394 \text{ rad/s}
		\end{bmatrix}
	$
	
	\item $\delta_m = 3 - \text{rank}[C_1(\beta_s(t))] = 2$
	\item $\delta_s = \text{rank}[C_{1s}(\beta_s(t))] = 0$
	\item $\delta_M = \delta_m + \delta_s = 2$
	\item DDOF is 2, the robot can move forwards/backwards, left/right. DOF is 3, the robot can move in the x, y, and yaw directions.
	\item
	\end{enumerate}
\end{document}