\documentclass{article}

\usepackage[margin=1in]{geometry}
\usepackage{amsmath}
\usepackage{amssymb}
\usepackage{enumitem}
\usepackage{color}
\usepackage{titlesec}

\setlist{nolistsep, noitemsep}

\titleformat{\section}[block]{\large\bfseries}{\thesection}{1em}{}
\titleformat{\subsection}[block]{\normalsize\bfseries}{\thesubsection}{1em}{}

\newcommand{\define}[1]{\noindent\textit{#1} | }

\begin{document}

\begin{center}
	\begin{large} \textbf{Programming Languages Final Exam} \end{large}
\end{center}

\hfill Charlie Coleman

\section{Functional Programming}

\subsection{What is a side effect, and why do functional languages not have them?}

Side effects are observable effects outside of the return of a function, like outputting text to the console or modifying a file. This makes formal verification of code easier, and allows the output of code to be more easily predicted.

\subsection{Give an example of how control flow is different in functional languages (particularly in Haskell). What type of statements in standard programming languages are not allowed in functional languages?}

There is no for loop or while loop in Haskell. Everything is executed using a function call.

\subsection{What is a higher order function? What is a first class object?}

\define{Higher order function} A function that takes another function as an argument or returns a function.

\define{First class object} An object with no restrictions on its use. It is an object that can be passed and returned to/by functions.

\subsection{How is I/O accommodated in functional programming languages, since it is pretty much purely based on side effects?}

I/O is only allowed in specific places within Haskell, like a main function or a larger I/O block

\subsection{What is a functor in Haskell?}

Functor is a type class in Haskell. It is designed to hold things that can be mapped over, like lists.

\subsection{How are types different in Haskell? Describe its type classes, and how they are different from object oriented classes.}

New data types can be defined as composites of existing types. Type classes define what operations can be performed on different data types.

\subsection{Be prepared to code Haskell functions, at the level of one of our homework assignments}

\subsection{What types of tools from functional programming are starting to show up in languages like Python and C++, and why are they being increasingly used there?}

\define{Map, filter, reduce} clean logic, parallelizable (especially map \& filter)

\define{First class objects} everything is an object in Python

\define{Higher order function} useful for callback functions and other use cases

\define{Lambda expressions} quick, single use functions prove useful in Python

\subsection{How are the tools in Python or C++ different from a more traditional functional language, like Python or Lisp?}

More complicated syntax, some require libraries to achieve the functionality, 

\subsection{Why do the more extreme proponents of functional programming object to structured paradigms? List a few examples from your reading on the subject (for that last essay).}

~

\section{Prolog}

\subsection{List a few applications of Prolog, or things that it can do well}

AI, Natural Language Processing, good for enumeration of all possible solutions to a problem

\subsection{What is unification, and how does Prolog attempt to do it?}

\define{Unification} applying the resolution principle and pattern matching things into appropriate spots

\noindent Rules of unification:
\begin{enumerate}
	\item A constant only unifies with itself
	\item Two structures unify if they have the same functor and arity, and the corresponding arguments unify recursively.
	\item A variable unifies with anything. If the other thing has a value, then the variable is instantiated. If the other thing is an uninstantiated variable, then the two are associated so that later values will be shared.
\end{enumerate}

\subsection{What is a functor in Prolog?}

A functor is the name outside the parenthesis of an operation or other data manip. If you say \verb|rainy(seatle)| rainy is the functor.

\subsection{How is a variable represented in Prolog? How are clauses formed?}

Variables are all caps. Clauses are what define the functor output. 

\begin{verbatim}
<base clause> ::= <structure>.
<non-base clause> ::= <structure :- <structures>.
\end{verbatim}

\subsection{Does the ordering of the clauses in a database matter in Prolog? Why or why not?}

Yes, the clauses are checked in the order they are typed.

\subsection{What is the cut (!) in Prolog?}

Cut prevents prolog from backtracking past a goal. This essentially locks variables in their current state.

\subsection{Again, be prepared to write or expand a short Prolog program, at the level of a single homework question or one of our examples from class.}

~

\section{COBOL}

\subsection{Name one place COBOL is still in use, and give several reasons.}

A lot of places. Many businesses maintain legacy COBOL. This is because the codebases are so large and still function, so many companies don't think it worthwhile to update.

\subsection{What are the principle strengths and weaknesses of COBOL?}

Easy to read syntax. Can handle large amounts of data. Reliable.

\noindent No pointers, user defined types, user defined functions (initially)

\subsection{How are COBOL programs organized?}

\begin{enumerate}
	\item Identification Division (required)
		
		| Supplies information about the program to the programmer and compiler
		
	\item Environment Division
	
		| Used to describe the environment in which program should run
		
	\item Data Division
	
		| Provides description to the data items the program will process
		
	\item Procedure Division (required)
	
		| Contains code used to manipulate the data
\end{enumerate}

\section{Educational Languages}

\subsection{What are some of the major innovations that show up in educational programming languages?}

Graphical representation of the code, syntax error prevention by only allowing valid commands

\subsection{What are a few programming languages that have been developed for teaching purposes, and what are some of their strengths and weaknesses?}

\define{Alice} 3D environment, use is limited mostly to games and video creation.

\define{Scratch} 2D environment, no first class functions, limited file I/O, can interact with Mindstorm

\end{document}