\documentclass[12pt]{article}

\usepackage[utf8x]{inputenc}
\usepackage[english]{babel}
\usepackage[letterpaper, margin=1in]{geometry}
\usepackage[normalem]{ulem}
\usepackage{hyperref}
\usepackage{tabularx}
\usepackage{enumitem}
\usepackage{graphicx}
\usepackage{pgfgantt}
\newcommand{\urlNW}[2]{\href[pdfnewwindow=true]{#1}{#2}}

\graphicspath{{./images/}}

\begin{document}
	
\vspace*{\fill}
\begin{center}

\begin{LARGE}IEEE Region 5 Robotics Competition\end{LARGE}~\\~\\


\begin{large}
Charlie Coleman - facilitator 

Amy Guo

Heli Wang\\~\\
\end{large}

Under Guidance from: Dr. Roobik Gharabagi \& Dr. Kyle Mitchell\\~\\

1 March, 2019

\includegraphics[scale=0.3]{logo}


\end{center}
\vspace*{\fill}

\begin{center}\includegraphics[scale=0.5]{ieee} \hspace{1cm} \includegraphics[scale=0.15]{slu}\\~\\\end{center}

\noindent Keywords: Pick and Place Robot, IEEE, Robotics Competition, Optical Image Recognition, Mothership, Raspberry Pi, Claw
\thispagestyle{empty}
\pagebreak

\thispagestyle{empty}
\tableofcontents
\pagebreak

\cleardoublepage
\setcounter{page}{1}

\section{Executive Summary}

For our project, we are designing a robot that is capable of navigating a board with various obstacles, picking up labeled cubes, and placing them in corresponding slots in a ‘mothership’. Our plan is to build a 4-wheeled robot to navigate the course. In order to properly sort the cubes, we created an OCR (optical character recognition) algorithm to identify the letters on the top face of the cube. We used a pi camera and a raspberry pi. To avoid the obstacles present on the course, we will use the provided JSON file to find a course through the obstacles.

\section{Intro \& Background} 

The Institute of Electrical and Electronics Engineers (IEEE) Region 5 Student Robotics Competition is sponsored by the Region 5 IEEE Committee. Region 5 includes over 90 students branches in the central United States. \cite{ieeer5} Students with a IEEE membership can enter the contest where each team is challenged to build an autonomous robot that avoids obstacles and picks up lettered cubes to place in the corresponding lettered box. \cite{robcomp}

\section{Market, Social, \& Ethical}

Pick and place robots are frequently used to benefit manufacturers with ease and automation. The main benefits for these robots are speed and consistency. Pick and place robots can be used in factories for: assembly, packaging, bin picking, and inspection. The benefits of speed and consistency provide return on investment (ROI) and help in profitable outputs. \cite{pickandplace}

The social aspect of this competition will motivate other students at Saint Louis University to compete in the IEEE Region 5 Robotics Competition. Our work will be showcased in the poster symposium and the walls of McDonnell Douglas Hall. With our work and advertisement, we will encourage students in the Electrical and Computer Engineering department to increase IEEE membership and involvement. IEEE membership is a great way to stay up to date on current technological discoveries, amazing student networking, attending conferences, and other career development tools. \cite{memval}

Robotic ethics will not be a concern because this design will only interact with simulated environments. The course is pre-determined and controlled, so all decision making systems will be based off of predefined rules. Though our system is autonomous, machine learning is not used to control response in the system’s behavior.

\section{Design Parameters \& Specifications}
\subsection{Design Parameters}

\subsubsection{Robot}

Single or multiple robots may use to complete the task. All robots must fit in 1’X1’ and not weigh over 40 lbs altogether.

A finishing light is required to flash once the robot finishes that round. The finishing light must be placed in the highest point of the robot.

\subsubsection{Mothership}

0.25 inch thick oak plywood was used to cut into different kinds of pieces like 

\begin{enumerate}[noitemsep]
	\item 1 - 8 ½ inch x 13 ½ inch 
	\item 2 - 8 inch x 6 ¼ inch 
	\item 4 - 4 ½ inch x 1 ¼ inch 
	\item 4 - 2 ½ inch x 1 ¼ inch 
	\item 2 - 8 ½ inch x 1 ¼ inch 
	\item 6 - 8 inch x 2 inch
\end{enumerate}

Piece “a” has been painted in white. Our next step is to paint letters A-F with stencils in the 2.5 inch X2.5 inch in the center square.

\subsubsection{Obstacles \& Blocks}

To build obstacles, we have cut dowels into 15 2" long pieces. For each piece, we drilled a 5/8 inch hole in the center. We have painted these pieces with gray spray paint. We still need to purchase the ping pong balls to place at the top of the dowel.

Blocks need to be sanded and paint them white. Use the stencil A-F to place a single letter onto the blocks and paint black. Our blocks and stencils have just arrived on Thursday, February 28th, so our next steps are to work on the blocks.

\subsubsection{Competition Board}

The competition board has been constructed. It has also been filled with wood-filler and sanded down to provide a smooth surface for our robot to navigate on. Lastly, it has been painted and dried.

\subsubsection{Corner Lights}

Corner lights still need to be worked on, but this is lower priority compared to other parts of our project. Four blue LEDs will be placed into the competition board, all facing the center.

\subsection{Specification}

\begin{itemize}[noitemsep]
	\item[-] Robot will have 10 mins before the round actually starts.
	\item[-] Once a round starts, no repair and changes can be made. 
	\item[-] Explosive and volatile liquid is banned in the robot.
	\item[-] Only wheeled, tracked or legged robots are allowed and especial one wheel has to keep in touch with the competition board
\end{itemize}

\section{Technical Analysis \& Recommendation}

	For our solution, we utilized a very simple chassis/driving mechanism for the robot to allow us to focus our efforts on the navigation \& OCR parts of the robot.

Although our design for the robot chassis was simple, we had several issues with sizing. To keep costs low, we used two motors with our four wheel design. We later decided to add motor encoders to tell us how much the motors have spun, so it could help us calculate our robot's location. We did not account for the addition area the encoders would add, so we had to minimize our wheel structure and place the gears on the outside perimeter to create enough room for our motor layout.

For our robot’s navigation, we could have used a secondary camera and used computer vision to identify and navigate around obstacles. We decided against this approach as it would increase the computation power needed in our onboard processor greatly. With the JSON file provided, we should be able to generate a 2D representation of the playing field and navigate the robot between obstacles fairly easily. 

In order to identify and place the cubes in their respective slots, we have decided to recognize the letter on the top face of the cube using OCR. This required the use of a camera and a decently powerful processor. We will need to provide ample light to the camera to get a good quality picture, so we will also need some form of lighting. 

During a preliminary testing, the OCR library we tested was able to correctly identify the letter in the image when it was turned less than 30 degrees off axis. This issue could be corrected in further testing by detecting the rotation of the image and correcting before attempting to find the letter within it. Currently our system is having trouble distinguishing between the letters "C" and "D" when the "C" is rotated twice or upside down. Our plan to correct this issue is to retake better quality images of real cases then rerun the algorithm.

After identifying the letter on the cube, we will need to move the cube to the mothership elsewhere on the playing field. We will do this using a claw designed to transfer the rotational movement of a motor to linear movement of the closing claw. By keeping the parts of the claw parallel as it closes, we are able to more easily keep the cube in the claw. The alternative to this, using a claw that pivots around a shared point, could push the cube forward instead of trapping in the claw.


\subsection{Budget Report}

\begin{table}[h!]
	\begin{tabular}{ll}
		\textbf{Item}              & \textbf{Cost}                          \\
		Chassis            & \$18.99                       \\
		2-wire motors      & \$29.98                       \\
		Motor controller   & \$19.98                       \\
		Steel plate        & \$4.99                        \\
		Wheels             & \$19.99                       \\
		Drive shaft        & \$5.49                        \\
		Wood cubes         & \$3.99                        \\
		Stencil            & \$2.99                        \\
		Total Amount Spent & $106.40 ($132.75 w. shipping)
	\end{tabular}
\end{table}


\section{Implementation Plan}

\begin{center}
\begin{ganttchart}
	[
		hgrid,
		vgrid,
		x unit=1.5cm,
		y unit title=0.8cm,
		y unit chart=0.8cm,
		milestone/.append style={xscale=0.4},
		bar/.append style={pattern=north east lines}
	]{1}{8}
	\gantttitle{Project Gantt Chart}{8} \\
	\gantttitle{2018}{4}
	\gantttitle{2019}{4}\\
	\gantttitle{Sept}{1}
	\gantttitle{Oct}{1}
	\gantttitle{Nov}{1}
	\gantttitle{Dec}{1}
	\gantttitle{Jan}{1}
	\gantttitle{Feb}{1}
	\gantttitle{Mar}{1}
	\gantttitle{Apr}{1} \\
	
	\ganttbar{Testing}{3}{8}\\
	\ganttbar{OCR}{1}{4}\\
	\ganttbar{Claw}{2}{5}\\
	\ganttbar{Chassis}{3}{6}\\
	\ganttbar{Power Supply}{5}{6}\\
	\ganttbar{Navigation}{4}{7}\\
	\ganttmilestone{Competition}{8}
\end{ganttchart}
\end{center}

\begin{enumerate}[noitemsep]
	\item Design \& build chassis -  Amy \& Heli
	\item Design \& build a claw - Amy
	\item OCR Development - Charlie
	\item Navigation Logic - Charlie
	\item Hardware Assembly - Amy \& Heli
	\item Power supply design \& implementation - Heli
\end{enumerate}

\section{References}
\begin{thebibliography}{4}
\bibitem{ieeer5}
	Robert Shapiro, IEEE Region 5 Website, 2018, http://ieeer5.org.
\bibitem{robcomp}
	IEEE Region 5 Robotics Competition, 2018, http://r5conferences.org/competitions/robotics-competition/.
\bibitem{pickandplace}
	Robotics Online Marketing Team, Pick and Place Robots: What Are They Used For and How Do They Benefit Manufacturers?, 03/13/2018, https://www.robotics.org/blog-article.cfm/Pick-and-Place-Robots-What-Are-They-Used-For-and-How-Do-They-Benefit-Manufacturers/88.
\bibitem{memval}
	IEEE, The Benefits of Membership, https://ewh.ieee.org/reg/3/IEEE\_member\_value.pdf.
	
\end{thebibliography}

\section{Appendix}

\subsection{Specifications}

\begin{itemize}[noitemsep]
	\item[-] Robot will have 10 mins before the round actually starts.
	\item[-] Once a round starts, no repair and changes can be made. 
	\item[-] Explosive and volatile liquid is banned in the robot.
	\item[-] Only wheeled, tracked or legged robots are allowed and especial one wheel has to keep in touch with the competition board
\end{itemize}

\subsection{Resources}

\subsubsection{Facilities}

\begin{itemize}[nolistsep, noitemsep]
	\item[-] Fabrication Lab
	\item[-] Senior Design Lab
	\item[-] Electronics Lab
	\item[-] Microprocessors Lab
\end{itemize}

\subsubsection{Lab Equipment}

\begin{itemize}[noitemsep]
	\item[-] Laser cutter
	\item[-] Digital Multimeter
	\item[-] Power Supply
	\item[-] Oscilloscope
\end{itemize}

\subsubsection{Computer Applications}

\begin{itemize}[noitemsep]
	\item[-] OpenCV
	\item[-] Tesseract OCR
	\item[-] Raspbian
\end{itemize}

\subsubsection{Specialized Hardware}

\begin{itemize}[noitemsep]
	\item[-] Raspberry Pi
	\item[-] Raspberry Pi Camera Module
	\item[-] Servo Motors
	\item[-] DC Motors
\end{itemize}

\subsubsection{Communication Protocols}

\begin{itemize}[noitemsep]
	\item[-] Universal Serial Bus
	\item[-] Camera Serial Interface
\end{itemize}

\subsection{Testing}
\subsubsection{OCR}
	\begin{enumerate}[noitemsep]
		\item Test using computer generated images based off stencil
		\begin{enumerate}[noitemsep]
			\item 1 image per letter rotated to various angles
			\item Run on lab computer
		\end{enumerate}
		\item Test using Raspbery Pi camera
		\begin{enumerate}[noitemsep]
			\item 10 images per letter
			\item Taken in well lit environment
			\item Run on lab computer
		\end{enumerate}
		\item Test using Raspberry Pi + camera
		\begin{enumerate}[noitemsep]
			\item 10 images per letter
			\item Taken on the assembled robot
			\item Run on the onboard computer (Raspberry Pi)
		\end{enumerate}
	\end{enumerate}
\subsubsection{Claw}
	\begin{enumerate}[noitemsep]
		\item Test on \& off the chassis
		\item Should pick up cubes with very high reliability
		\item Test 20+ times
	\end{enumerate}
\subsubsection{Navigation}
	\begin{enumerate}[noitemsep]
		\item Use assembled chassis, competition board, \& obstacles
		\item Verify the robot can reach any location a cube could be autonomously.
		\item Should be run with 5/10/15 obstacles (based on competition rules)
		\item Should be tested at least 10 times at each level
	\end{enumerate}
\subsubsection{Completed Robot}
	\begin{enumerate}[noitemsep]
		\item Match competition rules exactly
		\item Test 10+ times at each level of competition
		\item Record time and points as defined in the rules
	\end{enumerate}

\subsection{Personnel}

See following pages.
\end{document}