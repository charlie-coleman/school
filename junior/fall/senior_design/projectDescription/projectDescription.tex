\documentclass{article}

\usepackage{amsmath}
\usepackage{amssymb}
\usepackage[margin=1in]{geometry}
\usepackage{enumitem}
\usepackage{hyperref}

\title{ECE 4800 Project Description}
\author{Charlie Coleman}
\date{September 3rd, 2018}

\begin{document}
	
	\maketitle
	
	\noindent\textbf{IEEE 2019 Student Robotics Competition}
	
	The IEEE Robotics Competition for 2019 has not been announced yet, but we can look at previous years competitions to predict what will most likely be needed to compete in the 2019 competition. \href{http://r5conferences.org/competitions/robotics-competition/}{Last years competition} involved teams of up to 5 undergraduate students competing to find and sort colored disks autonomously. The playing field will contain 12 disks, 2 disks of each of the 6 different colors. These disks must be picked up by the robot and placed within the colored squares matching the disk color.
	
	This project would most likely involve designing a robot that would be able to navigate a course, identify colors, etc. all without human input. The team working on the project would have to identify requirements and choose appropriate materials/hardware to build the robot. A lot of the effort towards completion would involve programming the robot.
	
	This design would most likely incorporate some sort of microcontroller to interface with the sensors + motors within the robot. It may also involve a secondary microprocessor or microcomputer such as a Raspberry Pi to do more complex computation if it is deemed necessary. A chassis would need to be built for the robot. Motors and sensors would need to be chosen.
	
	I feel I would be able to contribute to the software portion of the design more so than the hardware, as I have taken courses in Controls and programming, and am currently taking a course in AI. I feel I could also help with hardware design and interoperability as I am currently taking Hardware Software Co-design. 
	
\end{document}