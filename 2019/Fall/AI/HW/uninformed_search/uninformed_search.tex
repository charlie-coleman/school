\documentclass{article}

\usepackage{amssymb}
\usepackage{amsmath}
\usepackage[margin=1in]{geometry}
\usepackage{enumitem}
\usepackage{hyperref}
\usepackage{tikz}
\usetikzlibrary{trees}

\begin{document}
	\begin{enumerate}
		\item[3.15:] \begin{enumerate}
			\item \begin{tikzpicture}[level/.style={sibling distance=60mm/#1}]
				\node {1}
					child {node {2}
						child{node {4}
							child{node {8}}
							child{node {9}}
						}
						child{node {5}
							child{node {10}}
							child{node {11}}	
						}
					}
					child{node {3}
						child{ node {6}
							child{node {12}}
							child{node {13}}
						}
						child{ node {7}
							child{node {14}}
							child{node {15}}
						}
					};
			\end{tikzpicture}
			\item \begin{enumerate}
				\item 1,2,3,4,5,6,7,8,9,10,11
				\item 1,2,4,8,9,5,10,11
				\item 1,2,3,4,5,6,7,8,9,10,11
			\end{enumerate}
			\item Would work well. From start: branch factor = 2, from goal: branch factor = 1
			\item Moving backwards from the goal is easier than finding the goal
			\item If goal is odd, subtract 1, divide by 2 until odd, repeat until you hit 1 (subtract 1 \& divide = move left from goal, divide by 2 means move right from goal). Play those moves backwards and you have your path.
		\end{enumerate}
	\end{enumerate}
\end{document}