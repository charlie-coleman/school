\documentclass{article}

\usepackage[margin=1in]{geometry}
\usepackage{url}
\def\UrlBreaks{\do\/\do-}
\usepackage{breakurl}
\usepackage[breaklinks]{hyperref}
\usepackage{titling}

\renewcommand\maketitlehooka{\null\mbox{}\vfill}
\renewcommand\maketitlehookd{\vfill\null}

\title{\textbf{Parallel and Serial Communication Channels\\ Computer System Design}}
\author{Charlie Coleman}
\date{03 May 2018}

\begin{document}
	\begin{titlingpage}
	\maketitle
	\end{titlingpage}
	
	\section{Serial AT Attachment (SATA)}
	
	\subsection{Overview of SATA} 
	\paragraph{} SATA is a bus that is meant to connect large storage medium, such as hard drives and solid-state drives, to a host device (computer). SATA is developed, maintained, and updated by the Serial ATA International Organization (SATA-IO), and was created in 2001 as a replacement for Parallel ATA \cite{satarev1}.
	
	\subsection{Reasons for Creation / Design Goals} 
	
	\paragraph{} SATA was designed to replace the previous generation of ATA connectors, the Parallel ATA (PATA). PATA used a 16 bit bus to transfer data, with a 133 MB/s bitrate at the end of its lifecycle. Serial ATA was designed to overcome any shortcomings of PATA. SATA achieved this through multiple changes in the design. Primarily, SATA uses 2 differential pairs of data lines, with a total of 7 conductors in the wire while PATA used 40-80 connectors \cite{pataspec}. This was an improvement as PATA cables were limited in their operating frequency and length. PATA cables could not be longer than 18 inches by spec, and operated in the 8MHz range. SATA cables have a maximum length of 1 meter, and they operate in the 1.5GHz range (in v1.0) \cite{satarev1a}. SATA v1.0 only had a maximum data transfer rate of 150 MB/s, which is only slightly faster than the last generation of PATA, but this would go up quickly with later generations.
	\paragraph{} Some other goals of the SATA design were low voltage operation, plug-and-play installation, minimal overhead, and competitive pricing (comparable to PATA).
	
	\subsection{SATA Uses}
	
	\paragraph{} SATA is used almost entirely inside of a computer to connect things like hard drives, solid-state drives, etc. to the host bus. There is also eSATA, which is short for external SATA. eSATA is meant to be a competitor with other external buses, like USB, and uses the same pins as SATA, but with a more robust connector.
	 
	\subsection{Unique Characteristics of SATA}
	
	\paragraph{} Similar to USB, the SATA specifications span many different form factors of connectors. The SATA connector is standard for 3.5" and 2.5" drives, but there are SATA connection types for a wide variety of devices. This includes Mini-SATA, which is for drives inside of portable devices such as a laptop, where the drive must be in a very small form factor. There is also SATA Express \cite{sataexpress}, which is a connection that supports SATA or PCI Express storage devices.
	 
	\section{Double Data Rate (DDR) SDRAM}
	
	\subsection{Overview of DDR}
	
	\paragraph{} The Joint Electron Device Engineering Council (JEDEC) developed the original DDR specification in 2000 as a successor to Single Data Rate (SDR) SDRAM. DDR is a memory type that two 64 bit words at a time, thus the "double" in the name. DDR is the standard for computer memory, and has gone through multiple generations, with the most current being DDR4.
	
	\subsection{Reasons for Creation / Design Goals}
	
	\paragraph{} DDR was developed to increase memory speeds in computers. By transferring two chunks of data per cycle, it could double the throughput of SDR SDRAM without increasing the clock speed, though clock speed was raised in some specifications. The first generation of DDR was capable of peak transfer speeds from 1600 MB/s to 3200 MB/s, well over the 800 MB/s that SDR was capable of \cite{ddrspec}.
	
	\paragraph{} DDR also lowered the power requirements of SDR. For SDR, $V_{CC}$ was 3.3V, the same as TTL \cite{sdrdata}. For DDR, $V_{CC}$ was 2.5V, meaning lower power consumption per bit. The operating voltage for DDR has gone down over the generations, and DDR4 now has a $V_{CC}$ of 1.2V \cite{ddr4}.
	
	\subsection{DDR Uses}
	
	\paragraph{} DDR SDRAM is used entirely for memory in computers. It has become the standard for memory in any type of computer.
	
	\subsection{Unique Characteristics of DDR}
	
	\paragraph{} The main differentiator between DDR and previous forms of SDRAM is the double data transfer that allows for substantial increases in speed. It is unique compared to its few (attempted) competitors with the large 64 bit channel and the ability to accept new commands before the previous has been completed. These features help to increase the speed over competitors.
	
	\pagebreak
	\begin{thebibliography}{9}
		\bibitem{satarev1} Serial AT Attachment Technical Specification v1.0 \\ \url{https://www.seagate.com/support/disc/manuals/sata/sata\_im.pdf}
		\bibitem{satarev1a} Serial AT Attachment Technical Specification v1.0a \\ \url{http://web.archive.org/web/20030213115847/http://www.serialata.org/collateral/zipdownloads/serialata10a.ZIP}
		\bibitem{pataspec} Parallel AT Attachment Interface with Extensions \\ \url{http://www.t13.org/Documents/UploadedDocuments/project/d0948r4c-ATA-2.pdf}
		\bibitem{sataexpress} SATA 3.2 includes SATA Express \\ \url{https://www.techspot.com/news/53567-sata-32-finalized-includes-sata-express-for-2-gb-s-of-bandwidth.html}
		\bibitem{ddrspec} Double Data Rate (DDR) SDRAM Specification \\ \url{http://cs.ecs.baylor.edu/\textasciitilde maurer/CSI5338/JEDEC79R2.pdf} 
		\bibitem{ddr4} DDR4 Information \\ \url{https://web.archive.org/web/20141010000932/http://www.corsair.com/\textasciitilde/media/Corsair/download-files/manuals/dram/DDR4-White-Paper.pdf}
		\bibitem{sdrdata} 512Mbit Single-Data-Rate (SDR) SDRAM Datasheet \\ \url{http://www.memphis.ag/fileadmin/datasheets/MEM51xxSDBATG\_10.pdf}
	\end{thebibliography}
\end{document}