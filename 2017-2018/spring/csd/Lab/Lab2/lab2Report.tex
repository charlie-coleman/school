\documentclass{article}

\usepackage{amsmath}
\usepackage{amssymb}
\usepackage[margin=1in]{geometry}
\usepackage{enumitem}

\title{Computer System Design Lab \# 2\\RS232 Signal Generation - Software}
\author{Charlie Coleman \\ Lab Partner: Amy Guo}
\date{January 28, 2018}

\newcommand{\Q}{\textbf{Q:}}
\newcommand{\A}{\textbf{A:}}
\newcommand{\sect}[1]{\noindent\textbf{#1}}

\begin{document}

\maketitle
\pagebreak

\noindent\textbf{Pre-Lab:} N.A.\\

\noindent\textbf{Objective:} The objective of this lab was to program a microcontroller in order to interact with a Bluetooth radio.\\

\noindent\textbf{Circuit Diagram:} N.A\\

\noindent\textbf{Outcome Predictions:} We will be able to successfully communicate with a Bluetooth device using the BlueFruit Radio. We will analyze the output signals using an oscilloscope to identify data.\\ 

\noindent\textbf{Equipment:}

\begin{itemize}[noitemsep, nolistsep]
	\item BlueFruit BLE Radio
	\item Oscilloscope
	\item PC Terminal
	\item C8051F320 microcontroller
\end{itemize}~

\noindent\textbf{Procedure:}

\begin{enumerate}[noitemsep, nolistsep]
	\item Connect the C8051F320 to the PC using the debugger
	\item Using the provided code and instructions, setup the C8051F320 
	\begin{enumerate}[noitemsep, nolistsep]
		\item Disable the watchdog timer so that the microcontroller does not restart regularly
		\item Need to implement the correct number of data bits, stop bits, set the parity bit
	\end{enumerate}
	\item Upload the code
	\item Connect the output of the C8051F320 UART to the PC UART to verify that the correct info is being sent
	\item Send various commands to the BlueFruit
	\begin{enumerate}[noitemsep, nolistsep]
		\item Rename the device to something easily recognizable
		\item Add some different variables, edit their values, etc.
	\end{enumerate}
	\item Check the Bluetooth connection on a cellphone to ensure the changes have gone into effect
\end{enumerate}~

\noindent\textbf{Recalculations and Predictions:} N.A\\

\noindent\textbf{Data and Observations:} The BlueFruit responded as expected. We were able to change the name on the BlueFruit easily so that the name would show up on our cellphones. When we added variables to the device and entered the value, we were able to see that info when using the nRF Connect app on a phone.\\

\noindent\textbf{Analysis \& Discussion:} No measurements or data was recorded during this lab. We were still able to monitor responses of the devices used in the lab, and those were the desired responses.\\

\noindent\textbf{Lab Questions:}

\begin{enumerate}[noitemsep,nolistsep]
	\item[\Q] How are PC baud rates historically generated?
	\item[\A] Baud clocks were generated by dividing the master clock in the system
	\item[\Q] How are baud rates generated in the C8051F320?
	\item[\A] The baud rate is given by the equation $\cfrac{T1_{clk}}{256-T1H} \times \cfrac{1}{2}$, where $T1_{clk}$ is the frequency of the clock supplied to timer 1, and T1H is the high byte of timer 1 (reload value)
	\item[\Q] What is the percent difference in the PC baud rate and the baud rate of your designs?
	\item[\A] The master clock on the C8051F320 is 12MHz, which can be divided evenly into 9600 baud. This means there is likely very minimal difference in PC baud and our design.
	\item[\Q] What is the maximum allowed difference in baud rates at the rate you choose, demonstrate your answer with a plot, annotation, and text.
	\item[\A] It is recommended to be within 1-2\% of the desired baud rate, so at 9600 baud, we can be off by about 96-192. If we sample at the middle of each bit, we will be able to sample 25-50 bits before the drift will cause losses. 
\end{enumerate}~

\noindent\textbf{Results:} Overall, the experiment worked as expected. We were able to observe changes in the BlueFruit configuration and look at the output of the C8051F320 using the PC.\\

\noindent\textbf{Conclusions:} This lab helped us to understand the configuration steps that are required to get multiple RS232 devices to communicate to each other. The lab helped us learn to use different software to implement code on microcontrollers, and familiarized us with different concepts like watchdog timers. These skills will be very helpful when we need to use any sort of microcontroller or UART in the future.

\end{document}